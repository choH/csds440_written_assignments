
\documentclass[12pt]{article}
\usepackage{times}
\usepackage{setspace}
\setstretch{1.5}
\usepackage{amsmath,amssymb, amsthm}
\usepackage{graphicx}
\usepackage{bm}
\usepackage[hang, flushmargin]{footmisc}
\usepackage[colorlinks=true]{hyperref}
\usepackage[nameinlink]{cleveref}
\usepackage{footnotebackref}
\usepackage{url}
\usepackage{listings}
\usepackage[most]{tcolorbox}
\usepackage{inconsolata}
\usepackage[papersize={8.5in,11in}, margin=1in]{geometry}
\usepackage{float}
\usepackage{caption}
\usepackage{esint}
\usepackage{url}
\usepackage{enumitem}
\usepackage{subfig}
\usepackage{wasysym}
\newcommand{\ilcode}{\texttt}
\newcommand{\p}{\partial}
\newcommand{\vphi}{\varphi}
\usepackage{etoolbox}
\usepackage{physics}
\usepackage{xcolor}
\patchcmd{\thebibliography}{\section*{\refname}}{}{}{}



\makeatletter
\renewcommand{\@seccntformat}[1]{}
\makeatother

\begin{document}



\title{\textbf{CSDS 440: Assignment 7}}

\author{Shaochen (Henry) ZHONG, \ilcode{sxz517} \\ Mingyang TIE, \ilcode{mxt497}}
\date{Due on 10/16/2020, submitted \textcolor{blue}{early} on 10/09/2020 \\ Fall 2020, Dr. Ray}
\maketitle


% % % % % % % % % % % % % % % % % % % % % % % % % % % % % % % % % %
% % % % % % % % % % % % % % % % % % % % % % % % % % % % % % % % % %
% % % % % % % % % % % % % % % % % % % % % % % % % % % % % % % % % %
\section{Problem 28}

% % % % % % % % % % % % % % % % % % % % % % % % % % % % % % % % % %
% % % % % % % % % % % % % % % % % % % % % % % % % % % % % % % % % %
% % % % % % % % % % % % % % % % % % % % % % % % % % % % % % % % % %
\section{Problem 29}

We try to show that the new $K = aK_1 + bK_2$ is a valid kernel as it compliants to the Mercer's conditions.

\begin{align*}
    K(x, y) &= aK_1(x, y) + bK_2(x, y) \\
    &= aK_1(y, x) + bK_2(y, x) \ \ \ \text{as $K_1$ and $K_2$ are valid kernels.} \\
    &= K(y, x)
\end{align*}

So $K$ is symmetry. Now suppose $\forall v \neq 0$, we have:

\begin{align*}
    v^T \cdot K v &= v^T (aK_1 + bK_2)v \\
    &= a\underbrace{(v^T K_1 v)}_{\geq 0} + b\underbrace{v(v^T K_2 v)}_{\geq 0}  \\
    &\geq 0
\end{align*}

So $K$ is also PSD. We may say $K$ is a valid kernel as both of the Mercer's conditions are met.

% % % % % % % % % % % % % % % % % % % % % % % % % % % % % % % % % %
% % % % % % % % % % % % % % % % % % % % % % % % % % % % % % % % % %
% % % % % % % % % % % % % % % % % % % % % % % % % % % % % % % % % %
\section{Problem 30}

For $K_1$ to be a valid kernel, we must have $K_1(x, y) = \vphi_1(x) \vphi_1(y)$ where $\phi$ is a non-linear mapping of its input. Due to the same principle, we also have $K_2(x, y) = \vphi_2(x) \vphi_2(y)$. Which gives us:

\begin{align*}
    K(x, y) &= a K_1(x, y) K_2(x, y) \\
    &= a( \vphi_1(x) \vphi_1(y) \cdot  \vphi_2(x) \vphi_2(y) \\
    &= a[(\sum_{i} \vphi_{1i}(x) \vphi_{1i}(y) ) \cdot (\sum_{j} \vphi_{2j}(x) \vphi_{2j}(y))] \\
    &= a \sum_{i} \sum_{j} \vphi_{1i}(x) \vphi_{1i}(y) \vphi_{2j}(x) \vphi_{2j}(y) \\
    &= \sum_{i} \sum_{j} (\sqrt{a} \  \vphi_{1i}(x) \vphi_{2j}(x) ) \cdot (\sqrt{a} \  \vphi_{1i}(y)\vphi_{2j}(y)) \\
    &= \vphi(x)\vphi(y)
        \begin{cases}
        \vphi(x) = \sqrt{a} \  \vphi_{1i}(x) \vphi_{2j}(x) \\
        \vphi(y) = \sqrt{a} \  \vphi_{1i}(y)\vphi_{2j}(y)
        \end{cases}
\end{align*}

As the above $\vphi$ is created upon multiplying a constant $\sqrt{a}$ with non-linear mapping $\vphi_1$ and $\vphi_2$, it is also a non-linear mapping; and $K$ is therefore a valid kernel.


% % % % % % % % % % % % % % % % % % % % % % % % % % % % % % % % % %
% % % % % % % % % % % % % % % % % % % % % % % % % % % % % % % % % %
% % % % % % % % % % % % % % % % % % % % % % % % % % % % % % % % % %
\section{Problem 31}

\begin{align*}
    K(x, y) &= (x \cdot y + c)^3 \\
    &= (xy)^3 + 3(xy)^2 c + 3(xy)c^2 + c^3 \\
    &= \sum_i \sum_j \sum_k (x_i y_i)(x_j y_j)(x_k y_k) + 3c \sum_i \sum_j (x_i y_i)(x_j y_j) + 3c^2 \sum_i \sum_j (x_i y_i)(x_j y_j) + c^3 \\
    &=  \sum_i \sum_j \sum_k (x_i x_j x_k)(y_i y_j y_k) + \sum_i \sum_j (x_i x_j \cdot \sqrt{3c})(y_i y_j \cdot \sqrt{3c}) \\
    &\ \ \ \  + \sum_i (x_i \cdot c\sqrt{3})(y_i \cdot c\sqrt{3})+ \sqrt{c^3}\sqrt{c^3} \\
    &= \vphi(x)\vphi(y)
\end{align*}

Where
\begin{align*}
    \vphi(a) &= [a_1 a_1 a_1, \dots, a_i a_j a_k, \dots, a_n a_n a_n, \\
    &\ \ \ \  a_1 a_1 \sqrt{3c}, \dots,\sqrt{2}  a_i a_j \sqrt{3c}, \dots, a_n a_n \sqrt{3c},\\
    &\ \ \ \  a_1 c\sqrt{3}, \dots,  a_i c\sqrt{3}, \dots, a_n c\sqrt{3}, \\
    &\ \ \ \  \sqrt{c^3} ]
\end{align*}

which is a non-linear mapping of input $a$. So we have showed $K(x, y) = \vphi(x)\vphi(y)$, we will continue to show that $K$ is also symmetric and PSD in \textit{Question 32}.

% % % % % % % % % % % % % % % % % % % % % % % % % % % % % % % % % %
% % % % % % % % % % % % % % % % % % % % % % % % % % % % % % % % % %
% % % % % % % % % % % % % % % % % % % % % % % % % % % % % % % % % %
\section{Problem 32}






\end{document}