
\documentclass[12pt]{article}
\usepackage{times}
\usepackage{setspace}
\setstretch{1.5}
\usepackage{amsmath,amssymb, amsthm}
\usepackage{graphicx}
\usepackage{bm}
\usepackage[hang, flushmargin]{footmisc}
\usepackage[colorlinks=true]{hyperref}
\usepackage[nameinlink]{cleveref}
\usepackage{footnotebackref}
\usepackage{url}
\usepackage{listings}
\usepackage[most]{tcolorbox}
\usepackage{inconsolata}
\usepackage[papersize={8.5in,11in}, margin=1in]{geometry}
\usepackage{float}
\usepackage{caption}
\usepackage{esint}
\usepackage{url}
\usepackage{enumitem}
\usepackage{subfig}
\usepackage{wasysym}
\newcommand{\ilcode}{\texttt}
\usepackage{etoolbox}
\usepackage{physics}
\patchcmd{\thebibliography}{\section*{\refname}}{}{}{}

\makeatletter
\renewcommand{\@seccntformat}[1]{}
\makeatother

\begin{document}



\title{\textbf{CSDS 440: Assignment 1}}

\author{Shaochen (Henry) ZHONG, \inlinecode{sxz517} \\ Mingyang Tie, \inlinecode{mxt497}}
\date{Due on and submitted on 09/11/2020}
\maketitle


% % % % % % % % % % % % % % % % % % % % % % % % % % % % % % % % % %
% % % % % % % % % % % % % % % % % % % % % % % % % % % % % % % % % %
% % % % % % % % % % % % % % % % % % % % % % % % % % % % % % % % % %
\section{Problem 1}

For a dice roll, let $A = \{1, 2\}$, $B = \{2, 3, 4\}$, and $C = \{1, 3\}$. We have  $P(A) = \frac{1}{3}$, $P(B) = \frac{1}{2}$, and $P(C) = \frac{1}{3}$.

Now we have:

\begin{align*}
    P(A, B) &= \{2\} = \frac{1}{6} = P(A)P(B) \ \ \text{Thus $A$ is independent of $B$.} \\
    P(A \mid C) &= \frac{ \{1\} }{\frac{1}{3}} = \frac{1}{2} \\
    P(B \mid C) &= \frac{ \{3\} }{\frac{1}{3}} = \frac{1}{2} \\
    P(A, B \mid C) &= \emptyset = 0 \ \neq P(A \mid C) \cdot P(B \mid C)
\end{align*}

And this proven the statement.

% % % % % % % % % % % % % % % % % % % % % % % % % % % % % % % % % %
% % % % % % % % % % % % % % % % % % % % % % % % % % % % % % % % % %
% % % % % % % % % % % % % % % % % % % % % % % % % % % % % % % % % %
\section{Problem 2}

We can will this problem of having two points $x_1$ and $x_2$, uniformaly distributed on a line with a length of $\sqrt{2}$, since this is the length of function $x+y = 1$ in interval $(0, 1)$ is $\sqrt{2}$). Let $x$ be a random varianle $\in[0, \sqrt{2}]$, the PDF of this $x$ would be:

\[ f(x) = \begin{cases}
      \frac{1}{\sqrt{2}} & x \in [0, \sqrt{2}]\\
      0 & \text{otherwise}
   \end{cases}
\]

To calculate the square distance, we have $D = (x_1 - x_2) ^ 2$ and we have $E[(x_1 - x_2)^2] = E(x^2_1 - 2x_1x_2 +  x^2_2) =  E(x_1^2) + 2E(x_1\cdot x_2) + E(x_2)^2$

For $x_1$ and $x_2$, since the placement of two points are independent, we have the joint PDF of $x_1, x_2$ to be:

\[ f(x_1,x_2) = \begin{cases}
      \frac{1}{\sqrt{2}} \cdot \frac{1}{\sqrt{2}} = \frac{1}{2} & x_1, x_2 \in [0, \sqrt{2}]\\
      0 & \text{otherwise}
   \end{cases}
\]

Since the square distance is $D = (x_1 - x_2) ^ 2$, its expected value is $E[(x_1 - x_2)^2]$, which is:

\begin{align*}
    E[(x_1 - x_2)^2] &=   \int^{\sqrt{2}}_{0} \int^{\sqrt{2}}_{0} (x_1 - x_2)^2 \cdot f(x_1, x_2) \cdot dx_1 dx_2 \\
    &= \frac{1}{2} \int^{\sqrt{2}}_{0} \int^{\sqrt{2}}_{0} (x_1^2 - 2 x_1 x_2 + x_2^2) \cdot dx_1 dx_2 \\
    &= \frac{1}{2} \int^{\sqrt{2}}_{0}  \frac{x_1^2}{3} - 2\frac{x_1^2}{2}x_2 + x_2 x_1 \Big|^{\sqrt{2}}_{0} \cdot dx_2\\
    &= \frac{1}{2} \cdot \frac{2}{3} \\
    &= \frac{1}{3}
\end{align*}

% % % % % % % % % % % % % % % % % % % % % % % % % % % % % % % % % %
% % % % % % % % % % % % % % % % % % % % % % % % % % % % % % % % % %
% % % % % % % % % % % % % % % % % % % % % % % % % % % % % % % % % %
\section{Problem 3}

% % % % % % % % % % % % % % % % % % % % % % % % % % % % % % % % % %
% % % % % % % % % % % % % % % % % % % % % % % % % % % % % % % % % %
% % % % % % % % % % % % % % % % % % % % % % % % % % % % % % % % % %
\section{Problem 4}


% % % % % % % % % % % % % % % % % % % % % % % % % % % % % % % % % %
% % % % % % % % % % % % % % % % % % % % % % % % % % % % % % % % % %
% % % % % % % % % % % % % % % % % % % % % % % % % % % % % % % % % %
% \section{References}
% \nocite{*}
% \raggedright
% \bibliography{references.bib}
% \bibliographystyle{plain}


\end{document}